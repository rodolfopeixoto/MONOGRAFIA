
% %% Prof. Ausberto S. castro Vera
% %% UENF-CCT-LCMAT -Ci\^{e}ncia da Computa\c{c}\~{a}o
% %% Adaptado de: http://www.udcsummary.info
% %% 24 de Agosto de 2018 


% \chapter{UDC - Universal Decimal Classification}  %% comentar isto para artigo

% \section{Universal Decimal Classification}

% \subsection{ O que \'{e} UDC?}

% A UDC \'{e} o principal esquema de classifica\c{c}\~{a}o multil\'{\i}ngue do mundo para todas as \'{a}reas do conhecimento e uma ferramenta sofisticada de indexa\c{c}\~{a}o e recupera\c{c}\~{a}o. \'{E} um sistema de classifica\c{c}\~{a}o altamente flex\'{\i}vel para todos os tipos de informa\c{c}\~{a}o em qualquer meio.

% Por causa de seu arranjo hier\'{a}rquico l\'{o}gico e natureza anal\'{\i}tica-sint\'{e}tica, \'{e} adequado para organiza\c{c}\~{a}o f\'{\i}sica de cole\c{c}\~{o}es, bem como para navega\c{c}\~{a}o e pesquisa de documentos. A UDC est\'{a} estruturada de tal forma que novos desenvolvimentos e novos campos de conhecimento podem ser prontamente incorporados. O c\'{o}digo em si \'{e} independente de qualquer idioma ou script em particular (consistindo de numerais ar\'{a}bicos e sinais de pontua\c{c}\~{a}o comuns) e as descri\c{c}\~{o}es de classes que o acompanham apareceram em muitas vers\~{o}es traduzidas.

% \subsection{ Quem utiliza UDC?}
% O esquema \'{e} de uso mundial e foi publicado no todo ou em parte em mais de 40 idiomas diferentes. \'{E} utilizado em servi\c{c}os bibliogr\'{a}ficos, centros de documenta\c{c}\~{a}o e bibliotecas em cerca de 130 pa\'{\i}ses em todo o mundo. Cole\c{c}\~{o}es de biblioteca indexadas por UDC podem ser encontradas em OPACs e bancos de dados da biblioteca.


% \subsection{0. SCIENCE AND KNOWLEDGE. ORGANIZATION. COMPUTER SCIENCE. INFORMATION. DOCUMENTATION. LIBRARIANSHIP. INSTITUTIONS. PUBLICATIONS}

% Tabelas Principais:

% 00 Prolegomena. Fundamentals of knowledge and culture. Propaedeutics

% 001 Science and knowledge in general. Organization of intellectual work

% 002 Documentation. Books. Writings. Authorship

% 003 Writing systems and scripts

% \textbf{004 Computer science and technology. Computing. Data processing}
% \begin{itemize}
%   \item \textbf{004.01/.08 Special auxiliary subdivision for computing}
%   \begin{itemize}
%     \item    004.01 Documentation
%     \item    004.02 Problem-solving methods
%     \item    004.03 System types and characteristics
%     \item    004.04 Processing orientation
%     \item    004.05 System and software quality
%     \item    004.07 Memory characteristics
%     \item    004.08 Input, output and storage media
%   \end{itemize}

%   \item \textbf{004.2 Computer architecture}
%   \begin{itemize}
%     \item 004.22  Data representation
%     \item     004.23 Instruction set architecture
%     \item     004.25 Memory system
%     \item     004.27 Advanced architectures. Non-Von Neumann architectures
%   \end{itemize}

%   \item \textbf{004.3 Computer hardware}
%     \begin{itemize}
%     \item004.3`1/`2 Special auxiliary subdivision for hardware
%     \begin{itemize}
%       \item 004.3`1   Production of computers
%       \item 004.3`2  Computer installations
%     \end{itemize}
%     \item 004.31  Processing units. Processing circuits
%     \item 004.32  Computer pathways
%     \item 004.33  Memory units. Storage units
%     \item 004.35  Peripherals. Input-output units
%     \item 004.38  Computers. Kinds of computer
%   \end{itemize}

%   \item \textbf{004.4 Software}
%     \begin{itemize}
%     \item 004.4`2/`6 Special auxiliary subdivision for software
%     \begin{itemize}
%       \item 004.4`2 Software development tools
%       \item 004.4`4 Programming language translation
%       \item 004.4`6 Runtime environment
%     \end{itemize}
%     \item 004.41 Software engineering
%     \item 004.42 Computer programming. Computer programs
%     \item 004.43 Computer languages
%     \item 004.45 System software
%     \item 004.49 Computer infections
%   \end{itemize}

%   \item \textbf{004.5 Human-computer interaction. Man-machine interface. User interface. User environment}
%         \begin{itemize}
%         \item 004.51 Display interface
%         \item 004.52  Sound interface
%         \item 004.55  Hypermedia. Hypertext
%         \item 004.58  User help
%         \end{itemize}

%   \item \textbf{004.6 Data}
%     \begin{itemize}
%     \item 004.62  Data handling
%     \item 004.63  Files
%     \item 004.65  Databases and their structures
%     \item 004.67  Systems for numeric data
%     \end{itemize}

%   \item \textbf{004.7 Computer communication. Computer networks}
%     \begin{itemize}
%     \item 004.71 Computer communication hardware
%     \item     004.72  Network architecture
%     \item     004.73  Networks according to area covered
%     \begin{itemize}
%       \item 004.738  Network interconnection. Internetworking
%     \end{itemize}
%     \item     004.75  Distributed processing systems
%     \item     004.77  General networking applications and services
%     \item     004.78  Online computing systems for specific use
%   \end{itemize}

%   \item \textbf{004.8 Artificial intelligence}

%   \item \textbf{004.9 Application-oriented computer-based techniques}
%     \begin{itemize}
%     \item 004.91   Document processing and production
%     \item 004.92  Computer graphics
%     \item 004.93  Pattern information processing
%     \item 004.94  Simulation
%   \end{itemize}


% \end{itemize}

% 005 Management

% 006 Standardization of products, operations, weights, measures and time

% 007 Activity and organizing. Communication and control theory generally (cybernetics). 'Human engineering'

% 008 Civilization. Culture. Progress


%% TCC - Monografia
%% Ci\^{e}ncia da Computa\c{c}\~{a}o - LCMAT - CCT - UENF, 2018
%% 

% ---
\chapter*[Conclus\~{a}o]{Conclus\~{a}o}                  \label{conclusao}
\addcontentsline{toc}{chapter}{Conclus\~{a}o}

% As conclus\~{o}es deste trabalho podem ser divididas em tr\^{e}s partes: sobre as dificuldades encontradas no desenvolvimento do TCC, sobre o que foi feito para atingir os objetivos, e sobre o que poder\'{a} ser feito no futuro para melhorar ou complementar a pesquisa na dire\c{c}\~{a}o deste trabalho.

% Sobre as dificuldades encontradas podemos mencionar as seguintes:
% Neste trabalho foi considerado o estudo de alguns aspectos te\'{o}rico-pr\'{a}ticos da engenharia de requisitos e
% Vale a pena observar que por raz\~{o}es pr\'{a}ticas desse trabalho, no estudo de caso n\~{a}o foram consideradas todos os.....

% A metodologia utilizada, facilitou a descoberta e a classifica\c{c}\~{a}o dos mesmos, pois foi observado que quanto mais

% Em futuros trabalhos, poder\~{a}o ser considerados tamb\'{e}m, ....
%\section{Dificuldades}

Durante o desenvolvimento deste trabalho de Monografia, a maior dificuldade talvez tenha sido na escrita e no entendimento do como mostrar o problema e a hipótese que se queria trabalhar.
%\section{Objetivos}
O objetivo deste trabalho de monografia foi atingido a partir da implementação computacional das duas arquiteturas apresentadas, tendo uma experiência prática para gerar os resultados em quanto a escalabilidade, manutenibilidade e uso de multi-tecnologias que serão mostrados a seguir. 
%\section{O que poderá ser feito para melhorar ou complementar o trabalho}
Em relação a uma melhoras deste trabalho, acredita-se necessário evoluir a arquitetura e testar outras formas como por exemplo \textit{performance}, comunicação via mensagens, assim como outras abordagens da arquitetura baseada em microsserviços. %, seria muito rico para complementar o trabalho. 

%\section{Resultados Finais}
Com a experiência do desenvolvimento da arquitetura em camadas e também da baseada em microsserviços podemos listar alguns pontos sobre ambas as arquiteturas.
%\subsection{Implementação da Arquitetura}
A implementação baseada em microsserviços se mostrou bem mais complexa e demorada, haja visto que você precisa construir diversas camadas, duplica código e sua implementação exige cautela. A monolítica é muito mais simples, pois tudo que você precisa está disponível em um só lugar.

%\subsection{Escalabilidade}
Em termos de escalabilidade ambas arquiteturas se comportam bem, porém com uma arquitetura baseada em microsserviços você tem a facilidade de escalar um determinado contexto que tenha se tornado um gargalo, o que gera também diminuição de custos. Uma arquitetura monolítica, você terá a necessidade de levantar a base toda do código, muitas vezes o gargalo é em uma determinada funcionalidade e mesmo assim, teríamos que levantar toda a base de código.

%\subsection{Manutenibilidade}
Já do ponto de vista da manutenção, a arquitetura monolítica mostrou-se ser um pouco mais complexa, visto que muitas vezes se faz necessário entender algumas regras mais abrangentes. A arquitetura de microsserviços se mostrou mais simples de evoluir e manter, haja vista que são contextos menores e se há uma falha naquele contexto, vamos direto na parte que está com \textit{bug}.

%\subsection{Multi tecnologia}

Há complexidade em adicionar novas linguagens de programação ou frameworks em uma arquitetura monolítica, haja visto que esse tipo de arquitetura não foi projetada com esse intuito, porém é possível. Com a arquitetura baseada em microsserviços ficou claro que é mais fácil de se desenvolver contextos em tecnologias que favoreçam a resolução daquele contexto, pois haverá pequenos projetos com complexidades menores para se manter. Por exemplo, na abordagem criada em microsserviços, a textit{API Gateway} foi implementada para um contexto específico, ser a porta de entrada das requisições e o mesmo ser o único sistema a se comunicar com os outros sistemas internos. Para a resolução desse problema o NodeJS foi a escolha, porém poderiamos utilizar o framework Ruby on Rails. 

%\subsection{Deploy}
Já em relação ao deploy também na arquitetura baseada em microsserviço foi mais rápido, porém mais complexa, pois são diversos sistemas. A vantagem da arquitetura em camadas é que, a mesma, é enviada uma única vez; como o sistema exemplo utilizado neste trabalho ainda é pequeno o deploy mesmo sendo mais demorado que a baseada em microsserviços, ainda sim é muito rápido.

%\subsection{Logs}
Um grande problema da arquitetura baseada em microsserviços são os logs descentralizados e precisamos aqui recorrer a um software de terceiros (qual?), mais um software que demanda manutenção. Com o monolítico não é necessário, visto que os logs já estão centralizados na aplicação.
%\subsection{Autorização e Autenticação}

Em relação à autorização e autenticação, a arquitetura monolítica é mais simples e rápida a implementação, pois no exemplo deste trabalho utilizamos uma gem chamada \textit{devise} a configuramos facilmente. Já no microsserviços há todo um contexto para se preocupar, mesmo com o Firebase nos facilitando de desenvolver a parte de autenticação.

%\subsection{tolerância a falhas}
A arquitetura monolítica se for bem projetada há possibilidade de projetar uma arquitetura dessas para ser ante falhas em termos de código e de forma mais manual. Mas de modo geral, caso deixe de funcionar uma parte da aplicação, normalmente será um efeito colateral, por exemplo o banco de dados por algum motivo ter um problema, a aplicação inteira sofre esse efeito.

A arquitetura baseada em microsserviços consegue lidar muito bem com esse problema, caso não seja a parte central como a de usuário, Autenticação, autorização ou a API Gateway no caso desse projeto. Como são partes centrais, elas sofrem efeitos em cadeia e você não terá acesso ao sistema. Mas caso seja um problema na parte de comentários e posts, não teríamos problema para acessar outras partes do software, pois iria carregar normalmente em nossa aplicação mobile.

Para essa comparação podemos analisar que depende muito do que estaríamos descrevendo sobre tolerância a falhas.

%\subsection{Resumo final}
De forma geral e para concluir este trabalho, é importante entender que não há melhor arquitetura, mas sim a que mais vai se adequar a aquele projeto e equipe. Diante de tantas informações que foi possível extrair desse projeto, acredito que o melhor é obter a compreensão que as arquiteturas quando bem estudadas e aplicadas de forma correta poderá diminuir e aumentar a produtividade da equipe em resolver problemas.


\section{Trabalhos Futuros}
Está faltando trabalhos futuros, ou tarefas ou testes que vc considere sejam importantes de serem realizados como continuação de seu trabalho.



% Há algumas vantagens quanto ao desenvolvimento, como toda base de código está disponível em um lugar só, basta executar baixar de um repositório git sua base de código e então terá todo seu projeto disponível e de fácil acesso. Como o projeto é único a base de logs também é única, o que representa uma grande vantagem, não necessitando de um software de de terceiros. 


% Há também desvantagens, em termos de manutenção como a base de código utilizada como exemplo, na base há diversas regras de negócio e possível e ocorreu durante o desenvolvimento alguns bugs que demoraram a ser resolvido, devido a base de código e a falta de teste inicialmente. Para escalar a aplicação se faz necessário replicar toda a base de código novamente, o que aumenta o custo dos recursos.

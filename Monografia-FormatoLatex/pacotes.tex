%% TCC - Monografia
%% Ci\^{e}ncia da Computa\c{c}\~{a}o - LCMAT - CCT - UENF, 2018
%% 



% ---
% Pacotes fundamentais
% ---

\usepackage[brazil]{babel}      % portugu\^{e}s brasileiro
\usepackage{graphicx}			% Inclus\~{a}o de gr\'{a}ficos
\graphicspath{{Imagens/}}      % Especifies a pasta onde as imagens est\~{a}o armazenadas
\usepackage{cmap}				% Mapear caracteres especiais no PDF
\usepackage{lmodern}			% Usa a fonte Latin Modern			
\usepackage[T1]{fontenc}		% Selecao de codigos de fonte.
\usepackage[utf8]{inputenc}     % Codificacao do documento (convers\~{a}o autom\'{a}tica dos acentos)
\usepackage{lastpage}			% Usado pela Ficha catalogr\'{a}fica
\usepackage{indentfirst}		% Indenta o primeiro par\'{a}grafo de cada se\c{c}\~{a}o.
\usepackage{color}				% Controle das cores
\usepackage{url}
\usepackage{here}               % para fixar posicionamento de figuras e tabelas [H]
\usepackage{ulem} 

% ---
\usepackage{amsmath}
\usepackage{xcolor}

\usepackage{boxedtheorem}
\usepackage{listings}		

\usepackage[many]{tcolorbox}
\usetikzlibrary{calc}
% ---
% Pacotes adicionais, usados apenas no \^{a}mbito do Modelo Can\^{o}nico do abnteX2
% ---
\usepackage{lipsum}				% para gera\c{c}\~{a}o de dummy text
% ---

% ---
% Pacotes de cita\c{c}\~{o}es
% ---
\usepackage[brazilian,hyperpageref]{backref}	 % Paginas com as cita\c{c}\~{o}es na bibl
\usepackage[alf]{abntex2cite}	% Cita\c{c}\~{o}es padr\~{a}o ABNT

% ---
% CONFIGURA\c{C}\~{O}ES DE PACOTES
% ---
% ---
% Configura\c{c}\~{o}es do pacote backref
% Usado sem a op\c{c}\~{a}o hyperpageref de backref
\renewcommand{\backrefpagesname}{Citado na(s) p\'{a}gina(s):~}
% Texto padr\~{a}o antes do n\'{u}mero das p\'{a}ginas
\renewcommand{\backref}{}
% Define os textos da cita\c{c}\~{a}o
\renewcommand*{\backrefalt}[4]{
	\ifcase #1 %
		Nenhuma cita\c{c}\~{a}o no texto.%
	\or
		Citado na p\'{a}gina #2.%
	\else
		Citado #1 vezes nas p\'{a}ginas #2.%
	\fi}%
% ---

% informa\c{c}\~{o}es do PDF
\makeatletter
\hypersetup{
     	%pagebackref=true,
		pdftitle={\@title},
		pdfauthor={\@author},
    	pdfsubject={\imprimirpreambulo},
	    pdfcreator={LaTeX with abnTeX2},
		pdfkeywords={PalavraChave1}{latex}{abntex}{abntex2}{trabalho acad\^{e}mico},
		colorlinks=true,       		% false: boxed links; true: colored links
    	linkcolor=blue,          	% color of internal links
    	citecolor=blue,        		% color of links to bibliography
    	filecolor=magenta,      	% color of file links
		urlcolor=cyan,
		bookmarksdepth=4
}
\makeatother


%----------------------------------------------------------------------------------------
%	ASCV Defini\c{c}\~{o}es
%----------------------------------------------------------------------------------------

\newcommand{\email}{\begingroup \urlstyle{tt}\Url}

\definecolor{verde}{rgb}{0,0.5,0}
\definecolor{roxo}{rgb}{0.5,0,0.5}
\definecolor{marron}{rgb}{0.5,0,0}

\def\fim{$\lozenge$}
\def\h#1{\hspace*{#1cm}}
\newcounter{exem}[chapter]
\newcounter{defin}[chapter]
\newcommand{\exemplo}[1][]{\addtocounter{exem}{1}
                \par \noindent
                {\textcolor{marron}{\sf Exemplo \thechapter.\theexem}}  \ }

\newcommand{\defin}[1]{\addtocounter{defin}{1} \par \noindent
                {\bf Defini\c{c}\~{a}o \thechapter.\thedefin} \ {\sf \textcolor{blue}{ #1}}\\ }
\newcommand{\sol}{\par {\it Solu\c{c}\~{a}o: \ \ }}
\def\mini#1{\hspace*{2cm}\begin{minipage}{12cm} #1 \end{minipage}}

%\newtheorem{teo}{Teorema}[chapter]
\newtheorem{lema}{Lema}[chapter]
\newtheorem{algo}{Algoritmo}
\newtheorem{coro}{Corolario}[chapter]


%---------------------------------- DEFINI\c{C}\~{A}O ------------------------------------------------------


\definecolor{myblue}{RGB}{0,163,243}

\tcbset{mystyle/.style={
  breakable,
  enhanced,
  outer arc=0pt,
  arc=0pt,
  colframe=myblue,
  colback=myblue!10,   %%% !20
  attach boxed title to top left,
  boxed title style={
    colback=myblue,
    outer arc=0pt,
    arc=0pt,
    top=3pt,
    bottom=3pt,
    },
  fonttitle=\sffamily
  }
}

\newtcolorbox[auto counter, number within=chapter]{definition}[1][]{
  mystyle,
  %colback=white,
  rightrule=0pt,
  toprule=0pt,
  title=Defini\c{c}\~{a}o~ \thetcbcounter,
  overlay unbroken and first={
      \path
        let
        \p1=(title.north east),
        \p2=(frame.north east)
        in
        node[anchor=west,
             font=\sffamily,
             color=myblue,
             text width=\x2-\x1]
        at (title.east) {#1};
  }
}
%----------------------------------
